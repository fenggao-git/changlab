\documentstyle[12pt]{article}
\title{Fractal Dewetting of an Adhesive Film Between\\ Two Separating Plates}
\author{\\\\
Sunil Kumar Thamida, Pavlo Takhistov and Hsueh-Chia Chang\\
Department  of Chemical Engineering\\
University of Notre Dame\\
Notre Dame, IN 46556}
%\pagestyle{empty}
\sloppy
\headheight 0pc
\textheight 50pc
\textwidth 39pc
\topmargin 0pc
\oddsidemargin 0pc
\evensidemargin 0pc
\marginparsep 0pc
\marginparwidth 0pc
\footskip 5pc
\begin{document}
\baselineskip24pt
\maketitle
\bigskip
\pagenumbering{arabic}
 
A dewetting cylindrical film between two separating plates is observed to
produce
fractal finger patterns that penetrate radially inwards.  In lieu of the
classical tip
splitting, this fractal is generated by successive shielding of alternating
fingers.  By
analyzing this key shielding dynamics via a forced two-dimensional
Hele-Shaw model with
global force balance, cumulative node number density of the fractal on a 
unit circle is shown
to depend on
the radial position $\hat{r}$ as $1/[1-d~ Bo^{1/2} \ln (\hat{r}/r_0)]$ 
where $d = 0.07$ is a
universal
number , $Bo = F/\sigma r_0$ is the bond number defined by the constant
separating force
$F$ and the initial film radius $r_0$.  The area of the film is shown
to decrease in time $\hat{t}$ as $|\hat{t}-(\hat{t}_c^\ast/2\ln 2)|^{1/4}$ 
where
$\hat{t}^*_c = \frac{3\pi \mu
r^4_0}{8Fh^2_0}$
is the plate detach time of an axisymmetrically dewetting film with $h_0$
the initial
plate separation.  Fractal dewetting of a cylindrical film hence reduces
the adhesiveness
of the film by a universal factor of $\frac{1}{2\ln 2}$ or 0.72.
\newpage

\section{Introduction}

When an adhesive tape is peeled off, the adhesive gel between it and the
solid surface
often breaks into distinctive fractal patterns$^1$.  The mechanism behind
this fractal
formation still escapes us, even though there have been numerous
speculations, and its
effect on the adhesiveness of the tape not well scrutinized.  Hele-Shaw
flow also produces
fractal structures (see, for example, the collected work in Pelce$^2$)
when air is
pushed against the confined liquid, either mechanically or by lifting the
top plate at an
angle, or when liquid is injected radially outward from a point source in
the Hele-Shaw
cell.  Such fractals are better understood and they are known to be created
by a tip
splitting mechanism that produces finer and finer higher-generation 
fingers$^3$.  Some
shielding is also observed when fingers actually stop and are surrounded
and overtaken by
their neighbors.  Nevertheless, the tip-splitting mechanism is the
predominant one in
these classical fractals and it remains the focus of many active
studies--particularly the
use of conformal map techniques to track poles responsible for tip
splitting and cusp
formation (Tanveer$^4$).

In this study, we examine a fractal related to but different from both of
the above
systems.  As in the peeloff problem, two plates separate in our system with
the bottom one
being pulled by a constant gravitational force $F$.  Instead of the
viscoelastic adhesive,
however, we place a thin cylindrical film of very viscous Newtonian wetting
fluid (castor
oil) between the separating plates.  Fingers appear at the perimeter of the
film as in
Hele-Shaw fingering.  However, since the air fingers invade radially and
axisymmetrically
inwards and the plate separation increases with time, 
the fractal structure and
its formation mechanism are different from Hele-Shaw fingers.  The
predominant mechanism
is now almost entirely the shielding mechanism.  A subset of each
generation of fingers
are shielded by their neighbors and stop their radial advance inwards.
This shielding
mechanism occurs, on the average, to every other finger and is driven by
the expanding
finger width and the circular geometry.  Successive shielding occurs for
each generation
at specific radial positions that decrease geometrically.  Higher
generation fingers are
hence longer and larger in the hierarchy of this axisymmetric fractal.  We
experimentally
record the plate separation dynamics (the evolution of the separation
distance and the
detach time of the bottom plate)  and analyze the scalings of the skeleton
fractal
structure left on the top plate.  This skeleton consists of residue liquid
from the films
between the air fingers.  Where each finger stops produces a node in this
network of film
residues. Each node in the fractal is connected to exactly three 
neighboring nodes by thin liquid branches.We can easily 
estimate the cumulative 
number of nodes within a
certain radius.
This characterization of the fractal structure is measured for a wide range
of conditions.

Without appreciable contact-line resistance, our system can be described by
a uniformly
forced two-dimensional Hele-Shaw model with global force balance.  The
dimensionless
version of the resulting equations contains no parameter.  We exploit this
symmetry and
the approximate radial symmetry of the invading fingers which invade into
an axisymmetric
receding circle to obtain the pertinent scalings. Universal constants for
such scalings
are estimated from numerical simulation of renormalized versions of the
shielding
mechanism for each generation.  A universal correlation for the cumulative
node number
distribution is obtained that depends only on a Bond number, defined with
respect to the
constant force separating the two plates.  We also obtain a prediction
that, for all
conditions, the detach time with fractals is $0.72$ of that without fractals
in an
axisymmetrically dewetting film.  The latter detach time $\hat{t}^*_c$ is 
obtained explicitly from
an analysis of the governing lubrication equations.

\section{Experiments}

A schematic of our experimental set up is shown in Figure 1.  A square
Plexiglas plate of
area $900$ cm$^2$ is affixed to the top of four guiding Plexiglas rods of
diameter $2$ cm.  A
bottom plate with four circular holes roughly the size of and aligned with
the rods is then
glided along the guiding rods.  Friction between the moving plate 
and the guiding 
rods is minimized by polishing the contact surfaces.
A weight is attached to the bottom
plate to produce
the gravitational pull $F$.  The falling plate can tilt slightly from its
horizontal
position.  Consequently, repeated runs are often made to obtain
statistically accurate
cumulative node number distribution and plate separation dynamics.

Prior to each run, a small drop of castor oil (viscosity $\mu = 0.8$ Ns/m$^2$,
 surface tension $\sigma= 0.03$ N/m and density $\rho = 0.96 $ gm/cc) 
containing a red dye is placed on the
bottom plate.  The
static contact angle of Castor oil is about $10^\circ$ and it hence strongly
wets the plate.
A circular thin film appears when the two plates are
adjoined and the drop
expands radially as it is squeezed (see the first frame of Figure 2).  The
initial plate
separation, before the bottom plate is allowed to be pulled by the weight,
is typically
less than $1$ mm.  This value is determined from the projected 
area of the film and
the known volume of the liquid.  The
circular perimeter of the film consists of a tiny meniscus that is
concave radially outwards due to the wettability of castor oil.  
The liquid volume, initial
radius of the circular film, the initial plate separation and the applied 
force are recorded
precisely. The
weight is varied over a wide range in our experiments.

We record the fractal formation with a high-definition digital Kodakplus
camera focused from above the
top plate. In order to obtain a better picture of the interface
motion, we chose a white surfaced bottom plate and a transparent 
top plate with lights focused from the top. A red colored dye is mixed to the
castor oil in order to obtain a sharp image.  The camera typically can take a
frame every 
second and the images are sent to a dedicated Macintosh computer and 
processed by NIH$^6$ software.  A
particular sequence
of snapshots are shown in Figure 2.  The plate separation $\hat{h}(\hat{t})$ is
estimated from the projected area of the film and the initial 
volume of the liquid film since volume remains constant. After 
processing the image by the ADOBEPHOTOSHOP software, the area is 
estimated by counting the dark pixels using image analysis software 
in MATLAB$^7$.
We use the initial area of the film as a correlation
reference point for estimating the area of the film during 
successive time steps.  The recorded separation
dynamics for the
sequence in Figure 2 is shown in Figure 3a along with others.  Detachment
of the bottom
plate is seen to occur at a very specific and accurately estimatable detach
time $\hat{t}_c$.

The number of first generation fingers seen in Figure 2 is a weak function
of the initial
separation.  This indicates that the meniscus curvature plays a role but if
the initial
separation is sufficiently small, viz. if it does not vary over an
excessively wide
range, its effect is constant for all experiments.  Shielding of this first
generation
is apparent in the second snap shot, which also shows some shielding of the
second
generation.  Tip splitting is seen in one finger in this snap shot but
shielding is the
predominant mechanism behind the fractal.  By the third and fourth
snapshots, the
slight tilting of the plates 
that destroys the axisymmetry of the fingers begins to
adversely affect
the uniformity of the fingers and the larger fingers exhibit tip splitting more
readily.  Nevertheless, every other finger is still shielded on the average
even for
these higher generations, as is evident in Figure 2. The final 
pattern appears as a radially converging network.

The fractal fingers do not create new contact lines as the invading air fingers
are still
surrounded by the wetting castor oil.  In fact, the two initial circular
contact
lines on the perimeter remain pinned to the plates.  There is hence a
tremendous increase
in interfacial area as the fractal develops but little increase in the
length of the
contact lines.  This scenario breaks down only near $\hat{t}_c$ 
when large separation
necessitates true dewetting as the films surrounding the fingers rupture.
However,
contact-line resistance and true dewetting are not expected to
signifcantly affect
$\hat{t}_c$ and the evolution towards it for a highly wetting fluid.

The earlier generations of fine fingers are often destroyed as the fingers
widen and
coalesce at the late stages and the stable ones survive at the periphery of
the film-- 
contrast snapshots b and c in Figure 2.  There is also
some adjustment
of the surviving structures as the fingers advance radially.  We hence
askew from an
analysis of the evolving fractal in Figure 2.  Instead, we examine the
final skeletal
structure left on the top plate after the bottom one detaches.  A series
of patterns for different
weights are shown in Figure 4.  The imprints of the fingers are clearly seen in
these fully
developed fractals.  Despite their seemingly different structures, all
fractals in Figure
4 consist of radially expanding branches joined at three-way nodes where three
branches meet.
These nodes and branches are residue imprints of the fingers that stop at every
generation due to the
shielding mechanism.  We count the total number of nodes within a radius
$\hat{r}$ inside a
sector as shown in Figure 4 and divide it by the total number inside the
initial radius $r_0$ to obtain the
cumulative
node number distribution density $M(r)$.  We then ensemble average over all
sectors.  To
compensate for the tilting which destroys the axisymmetry, an average
$M(r)$ is obtained
from several runs for each condition.
The dependence of $M(r)$ and the detachment dynamics on the system conditions
is what we
seek in the following analysis.

\section{Forced Hele-Shaw Flow, Axisymmetric Dewetting and Shielding Dynamics}

The usual lubrication approximation for the constant-force adhesive problem
yields the
following forced Hele-Shaw equations$^8$ :

$$
\nabla^2 p = -\frac{16}{\pi a_0^2} \frac{da^2}{dt}\ \ \ \mbox {\ in\ } \Omega
\eqno(1a)
$$
$$
\frac{\pi}{a_0}\int_{\Omega}p d\Omega = -1
\eqno(1b)
$$
$$
\int_{\Omega} d\Omega = a
\eqno(1c)
$$
$$
u_n = - \frac{\pi a_0^2}{32a^2} \nabla_n p\ \ \ \mbox{\ on\ }\partial\Omega
\eqno(1d)
$$
$$
p = 0\ \ \ \ \mbox{\ on\ }\partial\Omega
\eqno(1e)
$$
where $\nabla^2$ is the two-dimensional Laplacian in the horizontal $x-y$
plane, $\Omega$
is the liquid region on the plane which is initially as a 
unit circle and $\partial\Omega$ the free-surface.
The dimensionless variables are the radial coordinate
$r = \hat{r}/r_0$, the pressure as $p = \hat{p}/(F/r_0^2)$, 
the area of $\Omega$
as $a = \hat{a}/r^2_0$, time as $t=\hat{t}/t_v$ and the 
velocity as $u = \hat{u}/(r_0/t_v)$. Here the symbols with hat are 
the dimensional variables, $r_0$ is the initial radius of the film, 
$F$ is the applied force,
 $t_v = 
\frac{3\pi \mu r^4_0}{8Fh^2_0}$ is the viscous dewetting time for 
an axisymmetric circular film, $h_0$ is the initial thickness of the film 
and $\mu$ is the viscosity of the liquid.  The forcing
term $\frac{da^2}{dt}$ in (1a) arises from the velocity gradient in the
vertical direction due to changing thickness
which is a function of time only for a spatially homogeneous separation
distance.  The
global force balance in (1b) accounts for the constant force of 
separation. In (1b) $a_0 = \pi$ for an initially circular film of unit radius,
but $a_0$ is retained in this equation in the anticipation of later 
analysis in a sector rather than the complete circle.
The free-surface
motion at
$\partial\Omega$ is determined by the normal gradient of the pressure in
(1d) as in any
Hele-Shaw lubrication approximation of the slot flow.  Due to the parabolic
lubrication
velocity profile for the transverse velocities, the mobility
proportionality constant
relating them to the pressure gradient scales as $h^2/12\mu$.  Due to mass
conservation and constant density assumption, $V_o=ah$ is
an invariant constant volume and hence the mobility coefficient in (1d) 
scales as $a^{-2}$ in
this separating-plates problem.  We have neglected surface tension, meniscus
effect and
contact-line resistance in the perimeter condition (1e).  There is hence no
parameter in
this dimensionless set of equations. We seek  
the interface shape $\partial \Omega (t)$ as a function of time.


If the perimeter recedes axisymmetrically, the perimeter $\partial \Omega$
is located at
$r = r^\ast$ and a simple integration of the inhomogeneous Laplace
equation yields an
inverted parabolic pressure profile
\begin{equation}
\addtocounter{equation}{1}
p(r,t) = \frac{16}{\pi a_0^2} 
\frac{da^2}{dt} \left[\frac{r^\ast(t)}{2}\right]^2
\left[1 -
\left(\frac{r}{r^\ast(t)}\right)^2\right]
\end{equation}
Since $\frac{da^2}{dt}$ is negative, the pressure decreases radially inward
and drives the
fluid towards the center to replenish the vertical fluid motion.  The
radial pressure
gradient increases linearly in $r$ due to the radial symmetry.

Inserting the pressure profile (2) into the force balance (1b) and invoking the
identity for
the perimeter radius and area (normalized with respect to their initial
values), one
obtains a simple relationship for the radial or area evolution of
axisymmetric dewetting film,
$$
r^\ast(t) = (1-t)^{1/8} 
\eqno(3a)
$$
$$
a^\ast(t) = a_0(1-t)^{1/4}
\eqno(3b)
$$
It can be inferred from (3a) or (3b) that 
complete dewetting occurs when $t=t_c=1$ 
in the dimensionless 
form or in terms of dimensional variables it is equal to $t_v$, or
\begin{equation}
\addtocounter{equation}{1}
\hat{t}^\ast_{c} = \frac{3\pi \mu r_0^4}{8Fh_0^2}
\end{equation}
Hence $\hat{t}_{c}^\ast$ is also 
the theoretical detachment time for the axisymmetric 
dewetting without fingers.
With simple mass conservation, one also obtains the plate separation from 
(3b) for
axisymmetric
dewetting,
\begin{equation}
(\hat{h}(\hat{t})/h_0) = [1-(\hat{t}/\hat{t}^\ast_c)]^{-1/4}
\end{equation}
In Figure 6, we compare $\hat{t}_c^\ast$ to the measured $\hat{t}_{c}$ 
and find a
significant
over prediction by $\hat{t}^\ast_c$.  However, when $\hat{t}^\ast_c$ 
in the blow-up
behavior of (5) is
replaced by the measured value in Figure 6, 
the separation evolutions of
Figure 3a do
collapse into a universal curve described by (5), as seen in Figure 3b.
This suggests
that flow within the center core, confined by the tips of the invading air
fingers, is
well-described by the axisymmetric flow despite the presence of
fingers. The fingering phenomenon only increases the velocity 
of receding interface by a constant fraction compared to the 
axisymmetric flow. We shall
exploit this observation to produce a scaling theory.

We also simulate numerically both axisymmetric
and fractal
dewetting with the forced Hele-Shaw model (1). 
The inhomogeneous Laplacian equation in (1a) is solved numerically by 
using PDETOOL of MATLAB$^7$ software. PDETOOL is a subroutine that solves 
partial differential equations in a bounded domain. During every time 
step, the PDE in (1a) is solved for the pressure field and the boundary 
is updated according to the normal pressure gradient  
as defined in (1d).
Simulation of axisymmetric
dewettig yields
a radial shrinkage in exact agreement with (3a).  We relax the radial
symmetry in an
attempt to simulate fractal formation.  The perimeter evolution shown in
Figure 7a is
obtained after 72 hours of computation.  The simulation begins with a
perfectly sinusoidal
distortion of a radially symmetric perimeter.  Despite such uniformity,
numerical noise is
sufficient to reproduce the shielding dynamics of the first generation of
fingers.  This
occurs for a range of initial finger width and amplitude we attempted. 
In our experiments, the tip of each shielded finger of this first 
generation slows down to a complete stop and seems to evolve towards a 
cusp before surface tension imposes a high-curvature tip, as seen in 
Figure 7b.
Cusp formation is also seen in our simulation without surface tension 
as shown in Figure 7c.
This is reminiscent of cusp formation of Hele-Shaw bubbles in a
two-dimensional extensional flow$^{4,5}$.  Here, the two advancing
neighbors
provide the extensional flow, which the shielded finger experiences.  However,
such cusp
formation requires enormous temporal and spatial resolution such that we
are not able to
extend the computation for all generations of fingers.  Nevertheless, Figure
7a clearly shows
the second-generation fingers advance radially inwards with a slight
azimuthal widening.
This width widening has already initiated the shielding dynamics of the
second generation.

Our inability to simulate numerically the entire pattern of fingers
necessitates a
different analysis of the shielding dynamics at each generation.  Assuming
self-similarity for each of the radially symmetric generations, an
idealized and
normalized shielding model involving only three fingers in a 
sector with periodic boundaries can be scaled to
describe shielding
events of every generation.

Consider a sector of unit radius and of angle $\theta_0$ as shown in Figure
8.  The initial
fingers at the perimeter are generated by a sinusoidal perturbation 
and by pinching the outer most points between two neighboring 
fingers. With this pinching we are able to simulate the 
initial fingers of particular wavelength observed in experiments.
  There is reflection symmetry across the radial boundaries
of the
sector.  This initial model configuration gives rise to fingers that penetrate 
inwards, converge, compete and result in shielding of alternate fingers 
as seen in the snap shots of Figures 2 and Figure 7 and
in our
simulation of Figure 8.


Simulation results of (1) with this initial configuration in the sector
model are shown
in Figures 7 and Figure 8. The governing equations described in (1) are 
modified for sector model with the initial area as 
$a_0 = \theta_0 /2$.
  Numerical noise is sufficient to induce the two
outside fingers to
first contract and then expand their widths.  The expansion then confines
the middle
finger which eventually stops at a specific radial distance $\Delta r_{0}$ 
from the initial radius of the film after an
interval of
time $\Delta t_{0}$.  The tip of this shielded fingers is stretched into 
a cusp as
its two
neighbors advance past it.  Since the governing equations (1) are without
parameters and
since the  initial finger configuration are precisely specified, the
shielding distance
$\Delta r_{0}$ and the shielding time $\Delta t_{0}$ are functions 
only of the sector angle
$\theta_o$ --
the only parameter in the problem.  A sequence of numerical 
simulations shown in
Figure 8 for
various $\theta_o$ yield a linear dependence as
$$
\Delta r_{0} = c_0\theta_0
\eqno(6a)
$$
$$
\Delta t_{0} = c_1\theta_0
\eqno(6b)
$$
where $c_0 = 2.0$ and $c_1 = 8.0$ as shown in Figure 9.  This linear dependence
is 
physically due to the fact that the liquid mass that must be displaced and
azimuthal distance
that must be covered for shielding scale linearly with respect to the
sector angle
$\theta_0$.
This is consistent with the earlier
observation that the
dependence of $\Delta r_{0}$ and $\Delta t_{0}$ on $\theta_0$ 
is due entirely to azimuthal
widening of the
advancing fingers and not their radial advance.

\section{Scaling Theories for Fractal Fingers}
The one-parameter generic shielding model of the
previous section and the simple $\theta _0$ scalings of the
shielding variables $\Delta t_{0}$ and $\Delta r_{0}$ allow us to
derive its pertinent scalings for all generations. Let $r _n$ be the
position of each generation, defined by when half of the
fingers stop and their tips form nodes while half of them advance forward and
expand laterally as shown in Figure 10b. Let 
$\theta _n$ be the corresponding sector angle for
each advancing finger of that generation as determined by the number of
fingers. We further assume that the number of fingers
reduces by a factor of $1/2$ for each generation. Hence, the sector angle of 
each generation doubles as 
$\theta_n = 2^n \theta_o$. 
The model equations in (1) still remain the same for each sector except for 
an appropriate sector area, $a_0 = \theta_n /2$ in the force balance (1b).
Since the ratio $a/a_o$ appears in (1a) and (1d), 
these equations remain same despite a change in the sector angle.
Hence the model equations in a sector of half angle $\theta _n$ during all 
generations of fingers can be formulated by (1a) to (1e).

%The governing equation for the pressure field can be written as
%\begin{equation}
%\addtocounter{equation}{1}
%\frac{1}{r} \frac{\partial}{\partial r}\left( r \frac{\partial p}{\partial r}
%\right) + \frac{1}{r^2} \frac{\partial ^2 p}{\partial \theta ^2} =
%\frac{-32 a}{\pi a_o^2} \frac{da}{dt}
%\end{equation}

As shown in the schematic 
diagram in Figure 10c, let $a=a_c + a_b$ where $a$ is 
the total area of the film, $a_c$ is the area of the film in the 
inner circular core  
part of the film and $a_b$ is the area of the film in the finger branches.
We shall capture the scalings for the flow dynamics of shielding events 
deep within the finger branches. Within this region, the flow is independent 
of the radial flow in the circular core and is driven mostly by the 
vertical motion around the fingers. We hence assume the flow and pressure 
field within the finger branches are driven only by $\frac{da_b}{dt}$. 
Moreover, we have observed that the film flow between the shielded finger and 
its neighbors is predominantly an azimuthal one responsible for the 
widening of the advancing neighbors. Hence omitting 
$\frac{\partial p}{\partial r}$ in (1a) and using $\frac{da_b}{dt}$ as
the driving force,
%Hence we could decouple the governing equation for the circular part and the 
%branches. Since there is no radial velocity in the branches, 
%$\frac{\partial p}{\partial r}$ is zero and hence the governing 
%equation in the branches becomes,
the pressure governing equation in the finger branches becomes,
\begin{equation} 
\addtocounter{equation}{1}
 \frac{1}{r^2} \frac{\partial ^2 p}{\partial \theta ^2} =
\frac{-32a}{\pi a_o^2} \frac{da_b}{dt}
\end{equation}
This modified governing equation mainly accounts for the thinning of the 
branches in the $\theta$ direction, viz. the widening of air
fingers responsible 
for the shielding dynamics. 
With these specifications and assumptions, the
fingers of every shielding event of every generation
can be mapped into the generic ones of the first generation as shown in 
Figure 10c and Figure 10d.
The scalings for the geometry coordinates are
$$
r \rightarrow r _n R \\
\eqno(8a)
$$
$$
\theta \rightarrow 2^n \Theta \\ 
\eqno(8b)
$$
Under these scalings, we first obtain the scaling for pressure and area 
from (1b) and (1c) as
$$
p \rightarrow r_n ^{-2} P \\
\eqno(9a)
$$
$$
 a \rightarrow 2^n r_n^2 A \\
\eqno(9b)
$$
The scaling for $a_b$, the area of the film in the branches would also 
be similar to (9b) as $a_b \rightarrow 2^n r_n^2 A_b$ since we integrate 
over $r$ and $\theta$ coordinates. By substituting these scalings for $r$, 
$\theta$, $p$, $a$ and $a_b$ in (7) we obtain the scaling for time $t$ as 
\begin{equation}
\addtocounter{equation}{2}
t \rightarrow 2^{2n} r_n^8 T 
\end{equation}
The above scalings in (8) and (10) give the scaling for velocity in the 
liquid film as
$$ 
u_\theta \rightarrow r_n^{-7} 2^{-2n} U_\Theta 
\eqno(11a)
$$
$$
u_r \rightarrow r_n^{-7} 2^{-2n} U_R
\eqno(11b)
$$
The normalization(8-10) maps (7) into
\begin{equation}
\addtocounter{equation}{1}
\frac{1}{R^2} \frac{\partial ^2 P}{\partial \Theta} =
\frac{-32 A}{\pi A_o^2} \frac{dA_b}{dT}
\end{equation}
where $R$, $\Theta$, $P$, $A$, $A_b$, $T$, $U_R$ and $U_\Theta$ are the 
normalized variables.
This is a universal governing equation that 
captures the self-similar 
shielding event that occurs between three adjacent fingers 
for all generations as shown in Figure 10d. 
It accounts for the important phenomenon of 
azimuthal widening of alternate fingers and shielding of the central finger. 
It gives us the important time scaling which would be utilized to 
establish the 
scaling for shielding time $\Delta t_n$ and shielding distance $\Delta r_n$ 
for each generation in terms of $\Delta t_{0}$ and $\Delta r_{0}$ of 
the first generation.
By substituting (11) in (1d) we obtain the scaled velocity for different 
generations $n$ as
$$
 U_\Theta = 2^n \left[\frac{\pi A_0^2}{2 A^2} \frac{\partial P}
{\partial \Theta} \right]
\eqno(13a)
$$
$$
 U_R = 2^{2n} \left[\frac{\pi A_0^2}{2 A^2} \frac{\partial P}
{\partial R} \right]
\eqno(13b)
$$
These scalings indicate that the velocity of the fingers in this 
scaled version during 
every generation is related to that of first generation as
$$
U_{\Theta , n} = 2^n U_{\Theta , 0} 
\eqno(14a)
$$
$$
U_{R,n} = 2^{2n} U_{R,0}
\eqno(14b)
$$
Since the shielding of the central finger occurs due to the azimuthal 
widening of the fingers, the shielding time and shielding distance 
for each generation in this scaled version due to (14a) and (14b) 
would scale as
$$
\Delta T_n = 2^{-n} \Delta T_0 
\eqno(15a)
$$
$$
\Delta R_n = 2^n \Delta R_0
\eqno(15b)
$$
where $\Delta T_0 = \Delta t_{0}$ and $\Delta R_0 = \Delta r_{0}$.
Hence, from the simulation of a single
shielding event in the first generation we can establish the shielding 
positions of all further generations. 


We seek first the radial locations of nodes of each
generation of fingers. Scaling (15b) and (8a) immediately imply
\begin{equation}
\addtocounter{equation}{3}
R_n -R_{n+1} = 2^n(R_0 - R_1) = 2^n \Delta r_{0}
\end{equation}
or in the unnormalized variables,
\begin{equation}
\frac{r_{n+1}}{r_n} = 1 - 2^n \Delta r_{0}
\end{equation}
where, from (6), the radial advance of the shielded
finger in the first generation is $\Delta r_{0} = c_0 \theta_0$
and $\theta _0 = 2 \pi/N_0$ is determined by the
initial number of fingers $N_0$ for the first
generation. Hence (17) is an 
iterative map for the fractal structure. 
In principle, if we know the number of initial fingers we can build the 
complete fractal structure from the node positions due to the 
shielding events.


This geometric relationship yields a fractal structure similar to 
the Cayley fractal tree$^9$ 
shown schematically in Figure 10a, which represents an idealized
axisymmetric model of the fractal skeletons of Figure 4.
The number of nodes in the structure can also be
obtained as a function of the initial number of
fingers $N_o$ or the sector angle $\theta _0$ they
define. Since shielding reduces the number of fingers 
by half for every generation, the number of such nodes for generation $n$ is
$N_n = N_0 2^{-n}$. Hence,
\begin{equation}
\frac{N_n - N_{n+1}}{r_n - r_{n+1}} = \frac{N_n/2}{r_n
2^n \Delta r_0} = \frac{N_n/2}{r_n\left
(\frac{N_0}{N_n} \right) \Delta r_{0}}
\end{equation}
As a result,
\begin{equation}
\frac{N_n - N_{n+1}}{r_n - r_{n+1}} = \frac{1}{N_0
\Delta r_{0}} \frac{N^2_n}{r_n}
\end{equation}
and with a continuum approximation, this equation can
be integrated to yield the node number distribution
\begin{equation}
\frac{N(r)}{N_0} = \frac{1}{\left[ 1 - \left(
\frac{\ln~r}{\Delta r_{0}} \right) \right]}
\end{equation}
The continuum approximation also yields a more
explicit expression for the node position of each
generation obtained in (17) as
\begin{equation}
r_n = \exp [\Delta r_{0} (1-2^n)]
\end{equation}
The cumulative node number density can
now be estimated,
\begin{equation}
\begin{array}{lll}
M(r_n)& =&\displaystyle \sum^{\infty}_{m=n} {N_m} /\sum^{\infty}_{m=0} {N_m} \\
 & = &\displaystyle\sum^{\infty}_{m=n} 2^{-m} N_n / \sum^{\infty}_{m=0} 2^{-m}{N_0} \\
 & =& N_n/N_0
\end{array}
\end{equation}
Upon combining (20) and (22) we obtain
\begin{equation}
M(r)=\frac{1}{\left[1-\frac{\ln r}{\Delta r_{0}}\right]}
\end{equation}
Hence, the dependence of $M(r)$ on the applied force $F$ 
enters through $\Delta r_0$ or the initial number of
fingers $N_0$ in the first generation. 

The initial
fingers are created by the classical Hele-Shaw viscous
fingering instability$^{10}$. For a planar interface, it has been 
estimated$^{11}$(Chouke, Van Meurs and Van der Poel)
that the selected wavelength $\lambda$ to
be $2 \pi h_0/(Ca)^{1/2}$, where $Ca=\Delta \mu u_r /\sigma$ is the 
Capillary number, $\Delta \mu \approx \mu _{oil}$ since the 
viscosity of oil is very high compared to that of air, $u_r$ is the velocity 
of the interface and $\sigma$ is the surface tension. Curvature effect
is negligible for the initial fingers and the planar front
approximation should be valid. Relating the
displacement velocity $u_r$ by the initial axisymmetric dewetting
speed from (3a), we obtain an estimate of the initial number 
of fingers as $N_0 = 2 \pi
r_0/\lambda = (F/3\pi \sigma r_0)^{1/2}$. However,
this theory omits capillary effects due to the
meniscus curvature. One hence expects $\sigma$ to be
larger than that of the pure fluid. Instead of
accounting for the meniscus curvature explicitly, we
obtain an empirical correlation for $N_0$ from our
experiments in Figure 11 by assuming that the scaling
from the classical linear stability theory remains
valid. This is in fact confirmed by our data which
yield
\begin{equation}
N_o = 0.87 \left (\frac{F}{\sigma r_o}\right)^{1/2}
\end{equation}
The number of fingers during the first generation is a function 
only of the Bond number, $Bo=F/\sigma r_o$. 
The constant 0.87 is universal as the initial plate
separation and meniscus curvature are roughly constant.
Combining (24) with (6a) and substituting for $\Delta r_0$ in terms of $Bo$ 
in (23) then yields a universal
correlation for cumulative node number density of the fractals
\begin{equation}
M(r) = \frac{1}{\left [ 1 - d~ Bo^{1/2} \ln (r) \right]}
\end{equation}
where $r=\hat{r}/r_0$, $d = 0.87/2 \pi c_0 = 0.07$ is a universal constant
and $Bo = (F/\sigma r_0)$ is the Bond number defined by the applied force, 
surface tension and the initial radius of the film. 
Correlation (25) is seen to accurately describe
all our measured $M(r)$ data in Figure 5.


Due to the fractal break up of the liquid film, the plates detach earlier
than predicted by the axisymmetric dewetting time $\hat{t}_{c}^\ast$ 
as shown in Figure 6. 
A similar derivation yields the correction to the
detach time due to fractal fingering. The detach time
is computed from the summation of time intervals spent 
during all generations of fingers. 
\begin{equation}
\begin{array}{llll}
t_{c} &=& \displaystyle \sum^{n_{max}}_{n=0} \Delta t_n \\
& = & \displaystyle \sum^{n_{max}}_{n=0} 2^{2n} r_n^8 \Delta T_n\\ 
%& = & \displaystyle \sum^{n_{max}}_{n=0} 2^{2n} r_n^8 2^{-n}\Delta t_{0}\\
& = & \Delta t_{0} \displaystyle \sum^{n_{max}}_{n=0} 2^{n} r_n^8
\end{array}
\end{equation}
where $\Delta t_n$ is the time interval between the
formation of generation $n$ and $n+1$ and is approximated
by scaling obtained in (10) and (15a).
Substituting (21) for $r_n$ and replacing the summation by an
integral from $r=0$ to $r=1$, one obtains
\begin{equation}
t_{c} = \frac{\Delta t_{0}}{\Delta r_{0}} \frac{1}{8 \ln 2}
\end{equation}
However, from the simulation results of first generation of fingers, 
$\Delta t_{0}$ and $\Delta r_{0}$
are proportional to $\theta _0$, the sector angle for the
first generation of fingers. By substituting these results from (6) in (27),
we obtain the detach time in the dimensionless form as
\begin{equation}
t_{c} = \frac{1}{2\ln 2} = 0.72
\end{equation}
%\begin{equation}
%\hat{t}_{c} = \frac{1}{2\ln 2} \left[\frac{3\pi \mu r_0^4}{8Fh_0^2}\right]
%\end{equation}
Hence, for all conditions, the fractal detach
time is equal to 0.72 times that of the axisymmetric
dewetting time 
according to our model,
\begin{equation}
 \frac{\hat{t}_c}{\hat{t}_{c}^\ast} = \frac{1}{2 \ln 2} = 0.72
\end{equation}
 This universal correlation is
in good agreement with our detach time data in Figure 6.


\section{Summary and Discussion}
Like all fractal objects, we have shown that
converging fractal Hele-Shaw fingers arise from a
single event that appears at every scale -- the
shielding of fingers by their neighbors. The
separating plates dictate that the fingers must widen.
This and their radially converging geometry then
require that some of the fingers must stop and give
way to others -- the physical reason for the
domination of shielding. The scaling invariance of the
governing equations allows us to scale away all parameters except
the sector angle $\theta _0$ to construct a
generic shielding model. With proper scaling based
on the approximation that the flow is predominantly
radially symmetric in the inner circle and azimuthal 
in the branches, captures all shielding events of
every generation as a function of $\theta _o$, the sector angle 
during the first generation of fingers. The key
parameter $\theta _o$ is then related to the initial
number of fingers $N_0$, which is accurately captured
by the scalings of classical linear theory for viscous
fingering. As such, all scaling properties of the
fractals can be modeled, resulting in the universal
correlation for the cumulative node number density and
a universal prediction for the detach time that are in
good agreement with experimental data.

\vspace{2cm}


\noindent Acknowledgment \newline
This work is partially
supported by an NSF grant ECS97-06873

\newpage
{\bf References}
\begin{itemize}

\item[]
$^1$Cuprien Gay and Ludwik Leibler, `` On Stickiness,'' Phys. Today, Nov, 
48-52 (1999).


\item[]
$^2$Pierre Pelce, ``Dynamics of Curved Fronts,'' Academic Press, Inc., 
pp 155-242 (1988). 

\item[]
$^3$L. M. Sander, P. Ramanlal and E. Ben-Jacob, ``Diffusion-limited 
aggregation as a deterministic growth process,'' Phys. Rev. A, {\bf 32}, 
3160-3163 (1985).


\item[]
$^4$S. Tanveer, ``Evolution of Hele-Shaw interface for small surface tension,''
Phil. Trans. R. Soc. Lond, {\bf 343}, 155-204 (1993). 

\item[]
$^5$Qing Nie and Fei Ran Tian, ``Singularities in Hele-Shaw flows,'' 
J. App. Math., {\bf 58}, 1, 34-54 (1998).

\item[]
$^6$We use NIH Image for Mac OS by Wayne Rasband, National Institute of 
Health available at http://rp.info.nih.gov/nih-image/ (NIH Image is a public 
domain software).

\item[]
$^7$We use MATLAB, version 5.3.0.10183, copyright 1999, The Mathworks, Inc. 
For product information visit http://www.mathworks.com.  

\item[]
$^8$D. J. Acheson, ``Elementary Fluid Dynamics,'' Calendron Press, Oxford, 
243-245 (1990).


\item[]
$^9$Manfred Schroeder, ``Fractals, Chaos, Power Laws,'' W. H. Freeman and Company, 
New York, 363-365 (1991).

\item[]
$^{10}$P. G. Saffman and G. I. Taylor, ``The penetration of a 
fluid into a porous 
medium or Hele-Shaw cell; containing a more viscous liquid,'' Proc. R. Soc. 
London, Ser. A, {\bf 245}, 312 (1958).

\item[]
$^{11}$R. L. Chouke, P. van Meurs and C. van der Poel, 
``The instability of slow, 
immiscible , viscous liquid-liquid displacements in permeable media,''
Pet. Trans. AIME, {\bf 216}, 188 (1959).


\end{itemize}



 \newpage
{\bf Figure Captions}
\begin{itemize}
\item[\bf Figure 1]
Schematic of experimental set up.
\item[\bf Figure 2]
Snapshot of the fractal evolution at $t = 0, ~ 1.5, ~
3.1,~4.7,\mbox{and}~ 6.4$ seconds (a to e). The initial
separation $h_0$ is 0.22 mm, the initial film radius
$r_0 = 6$ cm and the total weight corresponds to 19
Newtons.
\item[\bf Figure 3]
(a) The plate separation as a function of time for
a film of initial radius $r_0 = 6$ cm, initial thickness $h_0 = 0.22$ mm and 
for various applied Forces i)$F = 13.7$ N , ii) $F = 15.9$ N and 
iii)$F=19.0$ N .
\newline (b) Collapse of separation blow-up data by the
axisymmetric dewetting theory when the measured $\hat{t}_{c}$
is used.
\item[\bf Figure 4]
Typical skeletal fractals after detachment for a film of initial 
radius $r_0 = 6$ cm, initial thickness $h_0 = 0.22$ mm and an
applied force $F = 5.4,~11.2~13.7~\mbox{and} ~ 15.9 $ Newtons (a to d).
A typical sector used to
estimate $M(r)$ is shown in (b).

\item[\bf Figure 5] 
Comparison of cumulative node number density $M(r)$
predicted by our scaling theory 
with the experimental data.

\item[\bf Figure 6]
Comparison of the measured detach time $\hat{t}_c$ to the
axisymmetric dewetting prediction $\hat{t}^{\ast}_c$ and to
the corrected value 0.72 $\hat{t}_c^{\ast}$ from the scaling
theory.
\item[\bf Figure 7]
(a)Numerical simulation of fractal dewetting in a sector.
\newline (b)A shielding event from our images. 
\newline (c) Numerical
simulation of one shielding event within a sector
$(\theta_0 = 3^\circ)$. 
\newline Note that the shielded finger forms a
cusp when it stops.
\item[\bf Figure 8]
Shielding simulation for sectors of various angles, a)$2^\circ$ 
b)$3^\circ$ c)$4^\circ$ and d)$5^\circ$ . 

\item[\bf Figure 9]
The computed shielding distance $\Delta r_{0}$ and
shielding time $\Delta t_{0}$ as a function of sector angle $\theta _0$.

\item[\bf Figure 10]
(a)Schematic Cayley fractal tree structure.
\newline (b)Schematic of the hierarchy of the finger generations.
\newline (c)Schematic of the sector containing three neighboring fingers.
\newline (d)Schematic of the sectors of all generations scaled to 
  generic ones of the first generation.
\item[\bf Figure 11]
Correlation for the initial number of fingers
\end{itemize}

\end{document}




















